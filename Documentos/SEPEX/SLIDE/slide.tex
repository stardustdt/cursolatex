\documentclass[aspectratio=169,11pt]{beamer}
\usetheme{CambridgeUS}
\usepackage[utf8]{inputenc}
\usepackage[portuguese]{babel}
\usepackage[T1]{fontenc}
\usepackage{amsmath}
\usepackage{amsfonts}
\usepackage{amssymb}
\usepackage{graphicx}
\usepackage{times}
\author[ADAME, Eduardo]{Eduardo Adame Salles}
\title[Um exemplo de beamer]{Mini-curso de \LaTeX : Um exemplo de beamer}
%\setbeamercovered{transparent} 
%\setbeamertemplate{navigation symbols}{} 
%\logo{} 
\institute[CEFET/RJ]{Centro Federal de Educação Tecnológica Celso Suckow da Fonseca} 
%\date{} 
%\subject{} 
\begin{document}

\begin{frame}
\titlepage
\end{frame}

%\begin{frame}
%\tableofcontents
%\end{frame}
\section{Filosofia}
\begin{frame}{Filósofos Bacanas}
\begin{block}{Friedrich Nietzsche}
É difícil viver com as pessoas porque calar é muito difícil.
\end{block}
Essa frase não foi dita com essas palavras, uma vez que Nietzsche não falava Português.
\end{frame}

\section{História}
\begin{frame}{Historiadores Bacanas}
Essa frase é show!
\begin{block}{Leandro Karnal}
Quem respeita o governador e não respeita a faxineira não é um líder, e sim um interesseiro.
\end{block}
\end{frame}

\section{Matemática}
\begin{frame}{}
Quem descobriu o seguinte teorema?
\begin{block}{Teorema de ??}
$$a^2 = b^2 + c^2$$
Resposta: \pause{}Pitágoras
\end{block}
Lista de Cientistas:
\begin{itemize}
\pause{}
\item Newton
\pause{}
\item Euler
\pause{}
\item Bohr
\pause{}
\item Rutherford
\pause{}
\item Einstein
\end{itemize}
\end{frame}
\end{document}